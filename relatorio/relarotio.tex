\documentclass[12pt,letterpaper]{article}
\usepackage[utf8]{inputenc}	% Para caracteres en español
\usepackage{amsmath,amsthm,amsfonts,amssymb,amscd}
\usepackage[table]{xcolor} 
\usepackage[margin=3cm]{geometry}
\usepackage{ragged2e}
\usepackage{graphicx}
\usepackage{hyperref}
\usepackage{amssymb}
\usepackage{subfig}
\usepackage{alphalph}
\newlength{\tabcont}
\setlength{\parindent}{0.0in}
\setlength{\parskip}{0.05in}

\begin{document}
	
	\begin{table}[]
		\centering
		\label{my-label}
		\begin{tabular}{llll}
			\textbf{Nomes:}&Cainã Setti Galante &   \textbf{Nº USP:} & 10737115 \\
			& Rubens Gomes Neto &         & 09318484 \\
		\end{tabular}
	\end{table}
	
	\begin{center}
		\huge \bf
		Exercício-Programa II de MAC0210 \\
	\end{center}
	
	\section{Parte 0 - Laboratório}
	
	Nesta parte do problema tivemos que implementr as funções que servirão para o estudo da interpolação de polinômios aplicados a imagens. As decisões de projeto e detalhes da implementação estão descritos abaixo. 
	
	Vale lembrar que as funções trabalham com imagens com três canais de cores. Portanto, quando nos referirmos a um ponto de uma matriz, na verdade estaremos nos referindo a um vetor de três coordenadas, onde cada coordenada representa uma cor. As matrizes que usamos têm três dimensões.
	
	\subsection{compress.m}
	
	Esse arquivo contém apenas a função \texttt{compress}, com o seguinte protótipo:
	
	\begin{center}
		\texttt{function compress (originalImg, k)}
	\end{center}
	
	Ela recebe um arquivo de imagem no formato \textit{png} e devolve, em um outro arquivo \textit{png}, a imagem comprimida com a taxa $k$.
	
	A leitura da imagem recebida é armazenada em uma matriz grande. Comprimimos retirando todas as linhas e colunas $i$ tal que $i\%(k+1)=1$, onde $\%$ representa a operação de resto.
	
	A compressão é feita selecionando os pontos que possuem um par linha, coluna que satisfazem os requisitos, o atribuímos à matriz reduzida. O ponto $(x, y)$ da matriz grande é colocado no seguinte ponto da matriz reduzida:  $(\left \lfloor{\frac{x}{k+1}+1}\right \rfloor, \left \lfloor{\frac{y}{k+1}+1}\right \rfloor)$.
	
	Feito isso, a matriz pequena é escrita numa imagem \textit{png}.

	\subsection{calculateError.m}
	
	O arquivo possui apenas a função \texttt{calculateError}, a qual tem o protótipo:
	
	\begin{center}
		\texttt{function calculateError(originalImg, decompressedImg)}
	\end{center}
	
	Essa função calcula o erro relativo entre duas imagens usando as fórmulas fornecidas no enunciado (aqui, usamos para a imagem comprimida e uma imagem descomprimida).
	
	\clearpage
	
	Primeiramente, lemos as imagens e armazenamos em matrizes, então, fazemos a conta. O único detalhe da implementação é que usamos a função \texttt{norm} do Octave para calcular a norma euclidiana.
	
	\subsection{decompress.m}
	
	Esse é o arquivo mais importante dos três enviados. Ele faz a descompressão de uma imagem usando algum método de interpolação e nos devolve o arquivo com a imagem descomprimida. Ele possui as seguintes funções, com os protótipos:
	
	\begin{center}
		\texttt{function decompress (compressedImg, method, k, h)}
	\end{center}
	
	A função \texttt{decompress} recebe uma imagem em \textit{png} e a descomprime em uma razão $k$, utilizando o método bilinear ou o método bicúbico.
	
	Ela lê a imagem e a armazena em uma matriz. A descompressão será feita inserindo $k$ linhas e colunas entre as linhas e colunas da matriz. Calculamos o tamanho da $p$ da imagem descomprimida usando a fórmula $p = n+(n-1)\cdot k$, onde $n$ é o tamanho da matriz da imagem pequena.
	
	Dependendo do método escolhido, ela chama a função que desenvolverá o método da interpolação, que devolverá uma matriz com os pontos interpolados.
	
	Feito isso, essa matriz é escrita em um arquivo \textit{png}.

	\begin{center}
		\texttt{function B = bilinear (A, k, h, p)}
	\end{center}
	
	Esse representa um dos métodos de interpolação descrito no enunciado, o método Bilinear por partes.
	
	Para começar, chamamos a função \texttt{expande}, que nos devolve uma matriz com o tamanho que precisamos (mais detalhes abaixo).
	
	Com essa matriz definimos quadrados de lado $k+2$ e, então, armazenamos os vértices do quadrado. Usamos \texttt{X = inv(A)*B} para resolver o sistema linear do método da interpolação (no enunciado está na forma B = AX). Com a matriz X dos valores da solução, fazemos a interpolação para cada cor de todos os pontos dentro de cada quadrado definido, usando a fórmula dada:
	
	\begin{center}
		$f(x, y) \approx p_{ij} (x, y) = a_0 + a_1(x - x_i ) + a_2 (y - y_j ) + a_3 (x - x_i )(y - y_j)$
	\end{center}
	
	No entanto, fizemos pequenas adaptações para a implementação funcionar:
	
	\begin{center}
		\texttt{f = X(1) + X(2).*x + X(3).*y + X(4).*x.*y;}, onde:
	\end{center}
	
	\begin{itemize}
		\item \texttt{f} é o resultado do polinômio interpolador.
		\item \texttt{X(i)} é o correspondente a $a_{i-1}$, proveniente da solução do sistema linear.
		\item \texttt{x} e \texttt{y} são as coordenadas do ponto no quadrado, definidos assim:
		\begin{itemize}
			\item \texttt{x = ((m-i)/(k+1))*h;}, onde $m$ é a real coordenada do ponto na matriz, $i$ é o início do quadrado na matriz grande, $k$ é a taxa de descompressão e $h$ é referente ao lado do quadrado, definido no enunciado.
			\item \texttt{y = ((n-j)/(k+1))*h;}, onde $n$ é a real coordenada do ponto na matriz, $j$ é o início do quadrado na matriz grande, e o resto é análogo.
		\end{itemize}
		\item Isso faz com que os índices dentro do quadrado estejam entre 0 e $h$, e, por consequência, o quadrado tenha lado de tamanho $k$.
	\end{itemize}
	
	O cálculo é feito para todos os pontos que não são os vértices do quadrado. 
	
	Em nosso método, consideramos $(x_0, y_0)$ como $(1, 1)$ da matriz e $(x_{p-1}, y_{p-1})$ como $(p, p)$ da matriz.
	 
	\begin{center}
		\texttt{function B = bicubico (A, k, h, p)}
	\end{center}

	Essa função representa o outro método de interpolação descrito no enunciado, o método Bicubico.

	Da mesma forma que no outro método, chamamos a função \texttt{expande} que devolve uma matriz com o tamanho necessário (mais detalhes abaixo).

	Para o cálculo das derivadas parciais existem 3 funções: \texttt{derivax}, \texttt{derivay} e \texttt{derivaxy} que já verificam as condições para as derivadas na borda. Uma observação a se realizar é que consideramos que nos casos em que aparecem pontos que extrapolam a grade fizemos os cálculos utilizando a diferença unilateral com o ponto da borda e seu adjacente.

	Para cada quadrado $Q_{ij}$ descrito no enunciado calculamos a matriz com os 16 coeficientes de $p_{ij}$. Calculamos os valores das cores de cada pixel da imagem com a fórmula do polinômio interpolador para cada quadrado $Q_{ij}$ do enunciado.

	\begin{center}
		\texttt{function B = expande (A, k, p)}
	\end{center}
	
	Essa função faz algo parecido com o oposto do descrito em \textbf{compress.m}
	
	Recebe uma matriz quadrada $A$ de tamanho $p$ e coloca $k$ linhas e colunas de zeros entre suas linhas e colunas, atribuindo tudo isso em uma matriz $B$. 
	
	Vamos andando ponto a ponto, e caso esse ponto tenha o par (linha, coluna) = $(i, j)$ tal que ambos os números cumpram o requisito $x\%(k+1)=1$, onde $\%$ representa a operação de resto, o correspondente $(\frac{i-1}{k+1}+1, \frac{j-1}{k+1}+1)$ da matriz $A$ é atribuído na matriz $B$.
	
	
\end{document}
