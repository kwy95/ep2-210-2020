\documentclass[12pt,letterpaper]{article}
\usepackage[utf8]{inputenc}	% Para caracteres en español
\usepackage{amsmath,amsthm,amsfonts,amssymb,amscd}
\usepackage[table]{xcolor}
\usepackage[margin=3cm]{geometry}
\usepackage{ragged2e}
\usepackage{graphicx}
\usepackage{hyperref}
\usepackage{amssymb}
\usepackage{subfig}
\usepackage{alphalph}
\newlength{\tabcont}
\setlength{\parindent}{0.0in}
\setlength{\parskip}{0.05in}

\begin{document}

    \begin{table}[]
        \centering
        \label{my-label}
        \begin{tabular}{llll}
            \textbf{Nomes:}&Cainã Setti Galante &   \textbf{Nº USP:} & 10737115 \\
            & Rubens Gomes Neto &         & 09318484 \\
        \end{tabular}
    \end{table}

    \begin{center}
        \huge \bf
        Exercício-Programa II de MAC0210 \\
    \end{center}

    \section{Parte 0 - Laboratório}

    Nesta parte do problema tivemos que implementr as funções que servirão para o estudo da interpolação de polinômios aplicados a imagens. As decisões de projeto e detalhes da implementação estão descritos abaixo.

    Vale lembrar que as funções trabalham com imagens com três canais de cores. Portanto, quando nos referirmos a um ponto de uma matriz, na verdade estaremos nos referindo a um vetor de três coordenadas, onde cada coordenada representa uma cor. As matrizes que usamos têm três dimensões.

    \subsection{compress.m}

    Esse arquivo contém apenas a função \texttt{compress}, com o seguinte protótipo:

    \begin{center}
        \texttt{function compress (originalImg, k)}
    \end{center}

    Ela recebe um arquivo de imagem no formato \textit{png} e devolve, em um outro arquivo \textit{png}, a imagem comprimida com a taxa $k$.

    A leitura da imagem recebida é armazenada em uma matriz grande. Comprimimos retirando todas as linhas e colunas $i$ tal que $i\%(k+1)=1$, onde $\%$ representa a operação de resto.

    A compressão é feita selecionando os pontos que possuem um par linha, coluna que satisfazem os requisitos, o atribuímos à matriz reduzida. O ponto $(x, y)$ da matriz grande é colocado no seguinte ponto da matriz reduzida:  $(\left \lfloor{\frac{x}{k+1}+1}\right \rfloor, \left \lfloor{\frac{y}{k+1}+1}\right \rfloor)$.

    Feito isso, a matriz pequena é escrita numa imagem \textit{png}.

    \subsection{calculateError.m}

    O arquivo possui apenas a função \texttt{calculateError}, a qual tem o protótipo:

    \begin{center}
        \texttt{function calculateError(originalImg, decompressedImg)}
    \end{center}

    Essa função calcula o erro relativo entre duas imagens usando as fórmulas fornecidas no enunciado (aqui, usamos para a imagem comprimida e uma imagem descomprimida).

    \clearpage

    Primeiramente, lemos as imagens e armazenamos em matrizes, então, fazemos a conta. O único detalhe da implementação é que usamos a função \texttt{norm} do Octave para calcular a norma euclidiana.

    \subsection{decompress.m}

    Esse é o arquivo mais importante dos três enviados. Ele faz a descompressão de uma imagem usando algum método de interpolação e nos devolve o arquivo com a imagem descomprimida. Ele possui as seguintes funções, com os protótipos:

    \begin{center}
        \texttt{function decompress (compressedImg, method, k, h)}
    \end{center}

    A função \texttt{decompress} recebe uma imagem em \textit{png} e a descomprime em uma razão $k$, utilizando o método bilinear ou o método bicúbico.

    Ela lê a imagem e a armazena em uma matriz. A descompressão será feita inserindo $k$ linhas e colunas entre as linhas e colunas da matriz. Calculamos o tamanho da $p$ da imagem descomprimida usando a fórmula $p = n+(n-1)\cdot k$, onde $n$ é o tamanho da matriz da imagem pequena.

    Dependendo do método escolhido, ela chama a função que desenvolverá o método da interpolação, que devolverá uma matriz com os pontos interpolados.

    Feito isso, essa matriz é escrita em um arquivo \textit{png}.

    \begin{center}
        \texttt{function B = bilinear (A, k, h, p)}
    \end{center}

    Esse representa um dos métodos de interpolação descrito no enunciado, o método Bilinear por partes.

    Para começar, chamamos a função \texttt{expande}, que nos devolve uma matriz com o tamanho que precisamos (mais detalhes abaixo).

    Com essa matriz definimos quadrados de lado $k+2$ e, então, armazenamos os vértices do quadrado. Usamos \texttt{X = inv(A)*B} para resolver o sistema linear do método da interpolação (no enunciado está na forma B = AX). Com a matriz X dos valores da solução, fazemos a interpolação para cada cor de todos os pontos dentro de cada quadrado definido, usando a fórmula dada:

    \begin{center}
        $f(x, y) \approx p_{ij} (x, y) = a_0 + a_1(x - x_i ) + a_2 (y - y_j ) + a_3 (x - x_i )(y - y_j)$
    \end{center}

    No entanto, fizemos pequenas adaptações para a implementação funcionar:

    \begin{center}
        \texttt{f = X(1) + X(2).*x + X(3).*y + X(4).*x.*y;}, onde:
    \end{center}

    \begin{itemize}
        \item \texttt{f} é o resultado do polinômio interpolador.
        \item \texttt{X(i)} é o correspondente a $a_{i-1}$, proveniente da solução do sistema linear.
        \item \texttt{x} e \texttt{y} são as coordenadas do ponto no quadrado, definidos assim:
        \begin{itemize}
            \item \texttt{x = ((m-i)/(k+1))*h;}, onde $m$ é a real coordenada do ponto na matriz, $i$ é o início do quadrado na matriz grande, $k$ é a taxa de descompressão e $h$ é referente ao lado do quadrado, definido no enunciado.
            \item \texttt{y = ((n-j)/(k+1))*h;}, onde $n$ é a real coordenada do ponto na matriz, $j$ é o início do quadrado na matriz grande, e o resto é análogo.
        \end{itemize}
        \item Isso faz com que os índices dentro do quadrado estejam entre 0 e $h$, e, por consequência, o quadrado tenha lado de tamanho $k$.
    \end{itemize}

    O cálculo é feito para todos os pontos que não são os vértices do quadrado.

    Em nosso método, consideramos $(x_0, y_0)$ como $(1, 1)$ da matriz e $(x_{p-1}, y_{p-1})$ como $(p, p)$ da matriz.

    \begin{center}
        \texttt{function B = bicubico (A, k, h, p)}
    \end{center}

    Essa função representa o outro método de interpolação descrito no enunciado, o método Bicubico.

    Da mesma forma que no outro método, chamamos a função \texttt{expande} que devolve uma matriz com o tamanho necessário (mais detalhes abaixo).

    Para o cálculo das derivadas parciais existem 3 funções: \texttt{derivax}, \texttt{derivay} e \texttt{derivaxy} que já verificam as condições para as derivadas na borda. Uma observação a se realizar é que consideramos que nos casos em que aparecem pontos que extrapolam a grade fizemos os cálculos utilizando a diferença unilateral com o ponto da borda e seu adjacente.

    Para cada quadrado $Q_{ij}$ descrito no enunciado calculamos a matriz com os 16 coeficientes de $p_{ij}$. Calculamos os valores das cores de cada pixel da imagem com a fórmula do polinômio interpolador para cada quadrado $Q_{ij}$ do enunciado.

    \begin{center}
        \texttt{function B = expande (A, k, p)}
    \end{center}

    Essa função faz algo parecido com o oposto do descrito em \textbf{compress.m}

    Recebe uma matriz quadrada $A$ de tamanho $p$ e coloca $k$ linhas e colunas de zeros entre suas linhas e colunas, atribuindo tudo isso em uma matriz $B$.

    Vamos andando ponto a ponto, e caso esse ponto tenha o par (linha, coluna) = $(i, j)$ tal que ambos os números cumpram o requisito $x\%(k+1)=1$, onde $\%$ representa a operação de resto, o correspondente $(\frac{i-1}{k+1}+1, \frac{j-1}{k+1}+1)$ da matriz $A$ é atribuído na matriz $B$.

    \section{Parte 1 - Zoológico}

    Para os experimentos do zoológico, utilizamos as seguintes imagens:

    \begin{figure}[h]
        \subfloat[Figura 1\label{surreal1}]{
        \includegraphics[width=6.5cm,height=6.5cm]{zoo1.png}}
        \subfloat[Figura 1 em preto e branco.\label{surreal2}]{
        \includegraphics[width=6.5cm,height=6.5cm]{zoo1a.png}}
    \end{figure}

    \begin{figure}[h]
        \subfloat[Figura 2\label{surreal11}]{
            \includegraphics[width=6.5cm,height=6.5cm]{zoo3.png}}
        \subfloat[Figura 3\label{surreal21}]{
            \includegraphics[width=6.5cm,height=6.5cm]{zoo2.png}}
    \end{figure}

    \begin{itemize}
        \item Figura 1: gerada através da função
        $f(x,y) = (sen(x), \frac{sen(y)+sen(x)}{2}, sen(x)$, descrita no enunciado (281 por 281 pixels);
        \item Figura 2: gerada através da função $f(x,y) = (cos(x), 2\cdot cos(y), x^2)$ (281 por 281 pixels);
        \item Figura 3: gerada por $f(x,y) = (tan(x) - tan(y), tan(y)- y, tan(x))$ (505 por 505 pixels).
    \end{itemize}

    \clearpage

    \subsection{Usando o método Bilinear}

    Usando o método Bilinear, descomprimimos as imagens, e obtivemos as seguintes respostas para as perguntas do enunciado:

    \textbf{1.} Funciona bem para imagens preto e branco?\\
    Sim, pois comparando com uma imagem colorida, o erro não foi muito diferente.\\

    \textbf{2.} Funciona bem para imagens coloridas?\\
    Funciona. As imagens ficam visualmente parecidas e o erro fica pequeno (a maioria na ordem de $10^{-1}$)\\

    \textbf{3.} Funciona bem para todas as funções de classe $C^2$?\\
    Sim, como exemplificado pelas figuras 1 e 2.\\

    \textbf{4.} E para funções que não são de classe $C^2$? \\
    Também funciona, como na figura 3.\\

    \textbf{5.} Como o valor de h muda a interpolação? \\
    O valor de h não parece interferir tanto na interpolação, pois, como exibido abaixo pelas figuras (A) e (B), que foram descomprimidas com h diferente ($h = 5$ e $h = 7$), o erro ficou muito próximo.\\

    \textbf{6.} Como se comporta o erro?\\
    Os erros ficaram bem pequenos, como exibido abaixo. Foi interessante ver que mudando a cor de uma imagem seu erro muda, e que quanto menor o $h$ usado para compressão e descompressão, menor o erro. \\

    \textbf{7.} Considerando uma imagem de tamanho $p^2$, comprimida com $k = 7$ e descomprimida com $k = 7$ e com $k = 1$ três vezes, quais são as conclusões?\\
    As imagens ficam pareceidas visualmente, porém o erro é menor (como mostram as figuras (B) e (C)).

    Abaixo, seguem as imagens acima descomprimidas.

    \begin{figure}[h]
        \subfloat[1 descomprimida com h = 4 e k = 3, err = 0.76841\label{surreal121}]{
        \includegraphics[width=5cm,height=5cm]{zdec1.png}}
        \subfloat[1 descomprimida com h = 4 e k = 3, err = 0.77409 \label{surreal221}]{
        \includegraphics[width=5cm,height=5cm]{zdec1a.png}}
    \end{figure}

    \begin{figure}[h]
        \subfloat[2 descomprimida com h = 5 e k = 7, err = 1.0884 (FIGURA A)\label{surreal131}]{
            \includegraphics[width=5cm,height=5cm]{zdec2.png}}
        \subfloat[2 descomprimida com h = 7 e k = 7, err = 1.0884 (FIGURA B)\label{surreal231}]{
            \includegraphics[width=5cm,height=5cm]{zdec2a.png}}
    \end{figure}

    \begin{figure}[h]
        \subfloat[2 descomprimida com h = 2 e k = 1 três vezes, err = 0.09700 (FIGURA C)\label{surreal281}]{
        \includegraphics[width=5cm,height=5cm]{zdec2b.png}}
        \subfloat[3 descomprimida com h = 2 e k = 4, err = 0.66150\label{surreal181}]{
        \includegraphics[width=5cm,height=5cm]{zdec3.png}}
    \end{figure}

    Obs: As imagens foram comprimidas e descomprimidas com o mesmo k, para manter o tamanho e o cálculo do erro ser facilitado.

    \clearpage

    \subsection{Usando o método Bicúbico}

    Usando o método Bicúbico, descomprimimos as imagens, e obtivemos as seguintes respostas para as perguntas do enunciado:


    \textbf{1.} Funciona bem para imagens preto e branco?\\
    As imagens ficaram um pouco diferente das originais, perdeu um pouco do padrão da imagem original, mas ainda mantiveram padrões dos níveis de cor da imagem original. O erro não é exatamente muito grande pois ainda é próximo de 1. Se comparado com a descompressão para as imagens coloridas, o erro e o resultado estão bem parecidos. Um fator curioso é que se comparado com o experimento das imagens reais o erro é maior.\\

    \textbf{2.} Funciona bem para imagens coloridas?\\
    Em relação à cor, apenas uma teve uma grande diferença no padrão de cor em relação à original, e como nas imagens preto e branco o erro não seja tão distoante de 1. Em comparação com o experimento das imagens reais o erro é maior.\\

    \textbf{3.} Funciona bem para todas as funções de classe $C^2$?\\
    Mantiveram os padrões de organização dos elementos da imagem, embora em um dos testes o resultado foi uma imagem com cores muito diferentes.\\

    \textbf{4.} E para funções que não são de classe $C^2$? \\
    A imagem ficou bem diferente, a imagem perdeu parte do aspecto quadriculado e em todos os casos o erro ultrapassou 1.\\

    \textbf{5.} Como o valor de h muda a interpolação? \\
    Não tiveram diferenças significativas, conforme o h aumenta o erro aumenta.\\

    \textbf{6.} Como se comporta o erro?\\
    Ele diminui da imagem colorida para a preto e branco ainda que pouco, quase não muda quando se altera o h e diminui quando ocorre a compressão 3 vezes seguidas.\\

    \textbf{7.} Considerando uma imagem de tamanho $p^2$, comprimida com $k = 7$ e descomprimida com $k = 7$ e com $k = 1$ três vezes, quais são as conclusões?\\
    Na compressão em três vezes seguidas o erro diminuiu se comparado com a compressão com uma vez.

    Abaixo, seguem as imagens acima descomprimidas.

    \begin{figure}[h]
        \subfloat[1 descomprimida com h = 4 e k = 3, err = 0.85446\label{surreal12}]{
        \includegraphics[width=5cm,height=5cm]{zbicdec1.png}}
        \subfloat[1 descomprimida com h = 4 e k = 3, err = 0.82142 \label{surreal22}]{
        \includegraphics[width=5cm,height=5cm]{zbicdec1a.png}}
    \end{figure}


    \begin{figure}[h]
        \subfloat[2 descomprimida com h = 5 e k = 7, err = 1.0321\label{surreal13}]{
            \includegraphics[width=5cm,height=5cm]{zbicdec2a.png}}
        \subfloat[2 descomprimida com h = 7 e k = 7, err = 1.0322\label{surreal23}]{
            \includegraphics[width=5cm,height=5cm]{zbicdec2b.png}}
    \end{figure}

    \begin{figure}[h]
        \subfloat[2 descomprimida com h = 2 e k = 1 três vezes, err = 1.0208\label{surreal28}]{
        \includegraphics[width=5cm,height=5cm]{zbicdec2c.png}}
        \subfloat[3 descomprimida com h = 2 e k = 4, err = 0.88062\label{surreal18}]{
        \includegraphics[width=5cm,height=5cm]{zbicdec3.png}}
    \end{figure}

    \clearpage

    Obs: As imagens foram comprimidas e descomprimidas com o mesmo k, para manter o tamanho e o cálculo do erro ser facilitado.

    \section{Parte 2 - Selva}

    Para a selva, utilizamos as seguintes imagens:

    \begin{figure}[h]
        \subfloat[Figura 4\label{surreal14}]{
        \includegraphics[width=3cm,height=3cm]{1.png}}
        \subfloat[Figura 5.\label{surreal24}]{
        \includegraphics[width=3cm,height=3cm]{2.png}}
        \subfloat[Figura 6\label{surreal34}]{
        \includegraphics[width=3cm,height=3cm]{3.png}}
        \subfloat[Figura 7\label{surreal44}]{
        \includegraphics[width=3cm,height=3cm]{4.png}}
    \end{figure}

    \begin{itemize}
        \item Figura 4: Arara (colorida) (281 por 281 pixels);
        \item Figura 5: Cachorro (preto e branco) (505 por 505 pixels);
        \item Figura 6: Maçãs (colorida) (505 por 505 pixels);
        \item Figura 7: Bicicleta (preto e branco) (931 por 931 pixels).
    \end{itemize}

    \subsection{Usando o método Bilinear}

    Usando o método Bilinear, descomprimimos as imagens, e obtivemos as seguintes respostas para as perguntas do enunciado:

    \textbf{1.} Funciona bem para imagens preto e branco?
    Sim, pois comparando com uma imagem colorida de mesmo tamanho, o erro fica diferente, mas pequeno.\\

    \textbf{2.} Funciona bem para imagens coloridas?
    Também unciona. As imagens ficam visualmente parecidas e o erro fica pequeno (a maioria na ordem de $10^{-2}$)\\

    \textbf{3.} Como o valor de h muda a interpolação?
    Em imagens reais, o valor de h não parece interferir tanto na interpolação, pois, como exibido abaixo pelas figuras (F) e (G), que foram descomprimidas com h diferente ($h = 3$ e $h = 5$), o erro ficou muito próximo, mas parece diminuir quando $h$ aumenta.\\

    \textbf{4.} Como se comporta o erro?
    Os erros ficaram bem pequenos, como exibido abaixo. Foi interessante ver que mudando a cor de uma imagem seu erro muda, e que quanto menor o $k$ usado para compressão e descompressão, menor o erro, como mostrado pelas imagens (D) e (E). \\

    \textbf{5.} Considerando uma imagem de tamanho $p^2$, comprimida com $k = 7$ e descomprimida com $k = 7$ e com $k = 1$ três vezes, quais são as conclusões?\\
    As imagens ficam pareceidas visualmente, porém o erro é maior (como mostram as imagens (H) e (I)).

    Seguem as descomprimidas:

    \begin{figure}[h]
        \subfloat[4 descomprimida com h = 4 e k = 3, err = 0.057654 (FIGURA D)\label{surreal151}]{
        \includegraphics[width=5cm,height=5cm]{decompressed1.png}}
        \subfloat[4 descomprimida com h = 4 e k = 4, err = 0.065963 (FIGURA E)\label{surreal251}]{
        \includegraphics[width=5cm,height=5cm]{decompressed1a.png}}
    \end{figure}

    \begin{figure}[h]
        \subfloat[5 descomprimida com h = 3 e k = 7, err = 0.033480 (FIGURA F)\label{surreal161}]{
        \includegraphics[width=5cm,height=5cm]{decompressed2.png}}
        \subfloat[5 descomprimida com h = 5 e k = 7, err = 0.033479(FIGURA G)\label{surreal261}]{
        \includegraphics[width=5cm,height=5cm]{decompressed2a.png}}
    \end{figure}

    \begin{figure}[h]
        \subfloat[6 descomprimida com h = 2 e k = 7, err = 0.080604 (FIGURA H)\label{surreal171}]{
        \includegraphics[width=5cm,height=5cm]{decompressed3.png}}
        \subfloat[6 descomprimida com h = 2 e k = 1 três vezes, err = 0.093727 (FIGURA I)\label{surreal371}]{
        \includegraphics[width=5cm,height=5cm]{decompressed3b.png}}
    \end{figure}

        \begin{figure}[h]
            \subfloat[7 descomprimida com h = 5 e k = 5, err = 0.042874\label{surreal271}]{
                \includegraphics[width=5cm,height=5cm]{decompressed4.png}}
        \end{figure}

    \clearpage

    \subsection{Usando o método Bicúbico}

    Usando o método Bilinear, descomprimimos as imagens, e obtivemos as seguintes respostas para as perguntas do enunciado:

    \textbf{1.} Funciona bem para imagens preto e branco?\\
    Sim, as imagens ficam bem parecidas, apenas com aparência pixelizada e o erro é da ordem $10^-3$. O erro é ligeiramente menor para imagens em preto e branco quando comparado com as imagens coloridas.\\

    \textbf{2.} Funciona bem para imagens coloridas?\\

    Sim, as imagens ficam parecidas, com aparência pixelizada e o erro é da ordem $10^-3$. O erro é maior do que quando comparado com imagens preto e branco.\\
    \textbf{3.} Como o valor de h muda a interpolação?\\

    Apesar da variação não ser significativa, quando maior o h menor é o erro (como vista na imagem do pug).\\

    \textbf{4.} Como se comporta o erro?\\

    Sempre na ordem de $10^-3$, varia muito pouco conforme variamos o h, quando aumentamos o k o erro aumenta e no caso em que ocorre compressão 3 vezes o erro aumenta.\\

    \textbf{5.} Considerando uma imagem de tamanho $p^2$, comprimida com $k = 7$ e descomprimida com $k = 7$ e com $k = 1$ três vezes, quais são as conclusões?\\
    A compressão 3 vezes seguidas tem erro maior do que com 1 compressão.


    Seguem as descomprimidas:

    \begin{figure}[h]
        \subfloat[4 descomprimida com h = 4 e k = 3, err = 0.057363\label{surreal15}]{
        \includegraphics[width=5cm,height=5cm]{sbicdec1a.png}}
        \subfloat[4 descomprimida com h = 4 e k = 4, err = 0.067982 \label{surreal25}]{
        \includegraphics[width=5cm,height=5cm]{sbicdec1b.png}}
    \end{figure}

    \begin{figure}[h]
        \subfloat[5 descomprimida com h = 3 e k = 7, err = 0.033906\label{surreal16}]{
        \includegraphics[width=5cm,height=5cm]{sbicdec2a.png}}
        \subfloat[5 descomprimida com h = 5 e k = 7, err = 0.033904\label{surreal26}]{
        \includegraphics[width=5cm,height=5cm]{sbicdec2b.png}}
    \end{figure}

    \begin{figure}[h]
        \subfloat[6 descomprimida com h = 2 e k = 7, err = 0.079984\label{surreal17}]{
        \includegraphics[width=5cm,height=5cm]{sbicdec3a.png}}
        \subfloat[6 descomprimida com h = 2 e k = 1 três vezes, err = 0.091454\label{surreal37}]{
        \includegraphics[width=5cm,height=5cm]{sbicdec3b.png}}
    \end{figure}

    \begin{figure}[h]
        \subfloat[7 descomprimida com h = 5 e k = 5, err = 0.043302\label{surreal27}]{
            \includegraphics[width=5cm,height=5cm]{sbicdec4.png}}
    \end{figure}

\end{document}
