\documentclass[12pt, a4paper]{article}
\usepackage[headheight = 28pt, margin = 20mm, top = 30mm]{geometry}
\usepackage[brazil]{babel}
\usepackage[utf8]{inputenc}
\usepackage{fancyhdr}
\usepackage{graphicx}
\usepackage{subcaption}
\usepackage{listings}
\usepackage{amsmath}

\pagestyle{fancy}
\fancyhf{}
\lhead{MAC0210\\30/06/2020}
\chead{EP2\\}
\rhead{Rubens Gomes Neto\\9318484}

\begin{document}
\section*{Parte 1}
    Para essa parte foi necessário calcular uma função $g(x)$ em que valesse
    a seguinte propriedade $g(x^*) = x^* \iff f(x^*) = 0$, para $f(x) = e^x - 2x^2$
    foram testadas as seguintes funções:
    \begin{align*}
        g_{1}(x) &= \frac{e^x}{2x}\\
        g_{2}(x) &= \ln(2x^2)\\
        g_{3}(x) &= \frac{\sqrt{2e^2}}{2}\\
        g_{4}(x) &= x - \frac{f(x)}{f'(x)}
    \end{align*}
    Para haver convergência no método é necessário que $|g'(x)| \leq \rho$ tal que
    $\rho < 1$ próximo a raiz. Ao analisarmos os gráficos das funções $g_i$ vemos
    que apenas $g_{4}$ satisfaz essa condição em todas as raízes.\\
    \begin{figure}[h]
        \centering
        % \includegraphics[width=.7\textwidth]{funcs.png}
        \caption{funções parte 1}
    \end{figure}\\
    O programa implementado calcula as raizes da função $f(x)$ pelo método do ponto
    fixo, utilizando $g_{4}(x)$. Para tal é possível passar os limites $[a, b]$
    assim como o número de pontos para busca de intervalos suspeitos.\\

\section*{Parte 2}
    Para a segunda parte o programa criado calcula e gera uma imagem das bacias
    de Newton, com cores associadas as raízes e sombreamento associado ao número
    de iterações executadas até a conversão. Caso não convergisse até \emph{max\_iter}
    o ponto é associado ao preto.\\
    As funções testadas foram as seguintes:
    \begin{align*}
        f_{0}(x) &= e^{x} - 2x^2\\
        f_{1}(x) &= x^4 - 1\\
        f_{2}(x) &= x^3 - 1\\
        f_{3}(x) &= x^3 - 2x + 2\\
        f_{4}(x) &= \sqrt{x^3} - 2x\\
        f_{5}(x) &= e^{x^2} - e^{x}
    \end{align*}

    Para seleção de cores foi usada uma implementação do sistema \textbf{HSV} em
    que após contar e numerar as raízes cada raiz era associada a um ponto
    equidistante dos outros da banda de \emph{hue} usando a seguinte conversão
    pra \textbf{RGB} de acordo com a seguinte tabela:\\
    \begin{figure}[h]
        \centering
        % \includegraphics[width=.7\textwidth]{HSV-RGB-comparison.png}
        \caption[]{conversão HSV - RGB}
    \end{figure}\\
    Com isso algumas das imagens geradas foram as seguintes:
    \begin{figure}[h]
        \begin{subfigure}{.3\textwidth}
            % \includegraphics[width=.95\textwidth]{id1shd1.png}
            \caption{$f_{1}(x) = x^4 - 1$}
        \end{subfigure}
        \begin{subfigure}{.3\textwidth}
            % \includegraphics[width=.95\textwidth]{id2shd1.png}
            \caption{$f_{2}(x) = x^3 - 1$}
        \end{subfigure}
        \begin{subfigure}{.3\textwidth}
            % \includegraphics[width=.95\textwidth]{id3shd1.png}
            \caption{$f_{3}(x) = x^3 - 2x + 2$}
        \end{subfigure}
        \caption{Polinomiais com $l = -2 + 2i$; $u = 2 - 2i$}
    \end{figure}
    \begin{figure}[h]
        \begin{subfigure}{.5\textwidth}
            % \includegraphics[width=.95\textwidth]{id4shd1.png}
            \caption{shading = 1}
        \end{subfigure}
        \begin{subfigure}{.5\textwidth}
            % \includegraphics[width=.95\textwidth]{id4shd2.png}
            \caption{shading = 2}
        \end{subfigure}
        \caption{Função $f_4(x)$ com $l = -2 + 2i$; $u = 2 - 2i$}
    \end{figure}
    \begin{figure}[h]
        \begin{subfigure}{.5\textwidth}
            % \includegraphics[width=.95\textwidth]{id0shd2-025025075-025.png}
            \caption{$l = -0.25 + 0.25i$; $u = 0.75 - 0.25i$}
        \end{subfigure}
        \begin{subfigure}{.5\textwidth}
            % \includegraphics[width=.95\textwidth]{id5shd2-115005.png}
            \caption{$l = -1 + 1.5i$; $u = 0.5i$}
        \end{subfigure}
        \caption{(a): $f_{0}(x) = e^{x} - 2x^2$; (b): $f_{5}(x) = e^{x^2} - e^{x}$}
    \end{figure}

\end{document}
